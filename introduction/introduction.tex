\setcounter{chapter}{-1}
\chapter{Overview}
\mediumlinespacing
\hspace{10pt} The following document serves as the summary of my work during the past four years. Based on the conventional thesis structure, Chapter~\ref{ch:theory} presents a description of the main pieces forming the most complete particle physics theory, the Standard Model. This is  followed by a description of the idea for connecting the invisible final state of the Higgs Boson with the prospects for Dark Matter searches, formulated into Chapter~\ref{ch:Higgs_LHC_DM}. The two aforementioned chapters form Part I of this thesis, and serve as a theoretical/motivation basis, explained in my own words, allowing the reader to have a brief theory overview before moving forward with the details of the experimental approach. 

\hspace{10pt} Part II of this thesis serves as an introduction to the world of collider physics with the emphasis on the structure of the CMS experiment (Chapter~\ref{ch:cms_experiment}) as well as a description of its data acquisition system (Chapter~\ref{ch:daq}). Chapter 4 contains an overview of the Data Quality Monitoring system for the Level-1 trigger within the CMS experiment with an example being given in the form the Level-1 Trigger Calorimeter Layer 2. This section includes my work during the 2017-2018 period of data taking, when I was in charge of developing and maintaining that system. Chapter~\ref{ch:daq} also contains a detailed description of the implementation of High Level Trigger paths designed specifically for the purposes of the main study covered by this thesis. This part is serves as a summary of my work on this topic, as I was responsible for the design, development, implementation and testing of those paths.

\hspace{10pt} Part III presents the main results given in this thesis, the search for invisible decays of the Higgs boson at $\sqrt{s}=$~13~TeV. The focus of this thesis is the scenario in which the Higgs Boson is produced through the process of Vector Boson Fusion. Chapter~\ref{ch:objects} serves as an overview of algorithms deployed for the purpose of particle reconstruction followed with a discussion of respective object corrections (some of which were developed for this study specifically by the Imperial College analysis team). Chapters~\ref{ch:an_strategy} and~\ref{ch:control_regions} introduce the main strategy behind this study. Having been its lead analyser (and publication contact), these chapters showcase my work, performed as a member of the Imperial College analysis team. The overview of the treatment of uncertainties and the signal extraction strategy is the begins the discussion within the Chapter~\ref{ch:fit}, which culminates with a summary of final results that came out of this search. The discussion focuses on the benefits gained through the usage of the full dataset collected by the CMS detector during the 2017-2018 period and a novel approach to the analysis strategy with the usage of new triggers. Overall combination with the previously published results regarding the data collected during 2016 era is also presented yielding a preliminary legacy result for this study.

\hspace{10pt} Lastly, Chapter~\ref{ch:conclusion} serves as a conclusion of the journey written down in this thesis. It is used to present final statements on the main analysis presented with previous chapters as well as to indicate what the future holds. The prospects for the near future include the preliminary result of the combination of different searches for the invisible final state of the Higgs boson, where the previously described results focusing on the Vector Boson Fusions production mode are combined with results from studies targeting other hadronic production modes. These sections will summarise my work as a member of a UK wide collaboration of analysis teams forming the "Combined Higgs to Invisible Project" (CHIP) working group. Covering the prospects of the far future, a simulation study of future prospects for these analyses is presented, covering part of my work done in order to test the sensitivity of these processes and the behaviour of the upgraded CMS detector under expected future operation conditions of the Large Hadron Collider.


%The final few months of my PhD studies coincided with an unfortunate event which has affected the entire world. During the isolation period, courtesy of the global viral pandemic, seemed to showcase the falls of out system, 